\documentclass{article}
\usepackage{microtype}
\usepackage[T1]{fontenc}
\usepackage{libertine}
%\usepackage{anyfontsize}
\pagestyle{empty}
\linespread{1.2}
\begin{document}
\frenchspacing
\noindent
{\LARGE 6 Linux Libertine}\\
~\\
Linux Libertine is marketed by the Libertine Open Fonts Project as
a free and open source replacement for the proprietary typeface Times New
Roman, the default serif typeface on every Microsoft Windows platform since
1992. Like its rival, Libertine looks like the 19th century
book type, and includes features designed for modern use. It contains more
than 2000 Unicode characters and supports many different
languages. In \LaTeX, it behaves as any other expert font with oldstyle
numerals, true small caps, kerning and ligatures.
However, make sure to \textbackslash usepackage[T1]\{fontenc\} when using
this typeface, because Libertine doesn't set the font encoding by default.
%In general it's a good practice to always
%specify the font encoding (T1 for all the fonts on this page) yourself.

By the way, Libertine is the font used in the
{\fontsize{11.3}{0}\selectfont V\hspace{-4.6pt}V}\hspace{-.5pt}{\fontsize{11}{10}\selectfont\sc ikipedi}A
logo.

\sf The sans serif family (used in this paragraph) set by
\textbackslash usepackage\{libertine\} is called Biolinum. As you can
see, it's perhaps the most beautiful sans serif typeface available to
\LaTeX{} users. It looks a little like Zapf's Optima, due to
the application of subtle stresses that produce the suggestion of a
glyphic serif. Bio\-linum goes even further in this by featuring real
(although petite) serifs at the end of some strokes.

\end{document}
