\documentclass{article}
\usepackage{microtype}
\usepackage[T1]{fontenc}
\usepackage{kpfonts}
\pagestyle{empty}
\linespread{1.2}
\begin{document}
\frenchspacing

\noindent
{\LARGE 2 Kp-Fonts}\\
~\\
The ``Kp-Fonts'' from the \emph{Johannes Kepler project} are a neat, easy,
and complete replacement for the default Computer Modern font family.  By
using the package kpfonts (\textbackslash usepackage\{kpfonts\}) you replace
the default roman, sans-serif and monospace typewriter fonts of the complete
document, as well as the font used in the math sections.  The package has
many options, see the documentation in kpfonts.pdf.
%(on Unix-like systems,
%type \emph{texdoc kpfonts}).

The roman font-family of Kp-Fonts is a modified version of the open source
URW Palladio, a clone of the popular but
non-free Palatino font family created by Herman Zapf and first released
in 1948. 
%Named after the $16^{\textrm{th}}$ century Italian master of
%calligraphy Giovanni Battista Palatino, Palatino is based on the humanist
%fonts of the Italian Renaissance.
According to the creator of Kp-Fonts, Christophe Caignaert, the
modifications were made to give the font a more basic and dynamic shape.
The following table (taken from \mbox{kpfonts.pdf}) illustrates this.



\end{document}
